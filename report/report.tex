\documentclass[12pt, a4paper]{article}
\usepackage[margin=3cm]{geometry}
\usepackage{graphicx}
\usepackage{xeCJK}
\usepackage{indentfirst}
\usepackage[square,numbers]{natbib}
\usepackage{multirow}
\bibliographystyle{humannat}
\setcitestyle{authoryear,aysep={},notesep={:},open={(},close={)}}
\setCJKmainfont{Songti TC} 
\setCJKfamilyfont{weibei}{Weibei TC}
\setCJKfamilyfont{longsong}{CSong3HK}
\setCJKfamilyfont{heiti}{Heiti TC}
\setCJKfamilyfont{lantinghei}{Lantinghei TC}
\setCJKfamilyfont{kaiti}{Kaiti TC}

\newcommand{\weibei}[1]{{\CJKfamily{weibei}#1}}
\newcommand{\longsong}[1]{{\CJKfamily{longsong}#1}}
\newcommand{\heiti}[1]{{\CJKfamily{heiti}#1}}
\newcommand{\lantinghei}[1]{{\CJKfamily{lantinghei}#1}}
\newcommand{\kaiti}[1]{{\CJKfamily{kaiti}#1}}

\renewcommand*\contentsname{目錄}

\linespread{1.15}
\setlength{\parskip}{1em}
\setlength{\parindent}{2.5em}

\title{System Programming\\Programing Assignment 4 Report}
\author{電機三\ 金延儒\ B02901107}

\begin{document}
\maketitle
\section{測試}
測試講求的是準確,為了使測出來的時間更準確,我大概有以下幾個方面的改進。

首先是輸出的部分,一開始是把本來程式的output重導到/dev/null,但後來發現,即使導到/dev/null,依舊會多耗掉一些時間在user time。讓我們看看測資$n=5\times 10^7$時兩種的差別。其中箭頭代表時間隨著thread數的增加而增加($\nearrow$)、減少($\searrow$)或持平($\rightarrow$)。

\begin{center}
\begin{tabular}{l | l l r r r r r r}
\hline
(sec)& \multicolumn{2}{|l}{\# of threads} & 1 & 2 & 5 & 10 & 25 & 100 \\ \hline
\multirow{3}{*}{Delete output}
& real & $\searrow$ & 20.212 & 13.734 & 11.912 & 11.786 & 11.672 & 11.734 \\ \cline{2-9}
& user & $\rightarrow$ & 20.070 & 20.192 & 20.338 & 20.240 & 20.068 & 19.982 \\ \cline{2-9}
& sys & $\nearrow$ & 0.150 & 0.130 & 0.162 & 0.188 & 0.220 & 0.366 \\ \hline\hline
\multirow{3}{*}{To /dev/null}
& real & $\nearrow$ & 27.230 & 25.018 & 28.216 & 31.180 & 35.326 & 42.784 \\ \cline{2-9}
& user & $\nearrow$ & 27.078 & 30.944 & 37.132 & 40.962 & 45.812 & 53.208 \\ \cline{2-9}
& sys & $\nearrow$ & 0.168 & 0.192 & 0.254 & 0.282 & 0.440 & 0.840 \\
\hline
\end{tabular}
\end{center}

拿redirect to /dev/null 跟delete output 相比,我們可以觀察到system time雖然有增加,但是非常不顯著,最主要是user time部分的增加,又如果這部分增加的user time是可以平行化的,那麼均攤下來可能不會差非常多。但不幸的是,我們可以看到terminal輸出無法平行化,所以這會使得real time急遽增加,甚至原本因為threading帶來的效益也沒了。也就是說:在要輸出過程的情況中,threading在時間上的花費反而會更差,主因在於其中輸出的難以平行化。

再來是測資產生,餵給主程式的部分,如果我們將主程式的測資輸出到一個檔案裡面,測試時再由主程式直接去讀取那個檔案,這樣會造成一個問題:檔案在讀寫是要trap into system,會多花一些system time,並且大幅減少效率。這個的解決方法很簡單,就是把產生測資的程式直接pipe給merger,這樣子直接在記憶體之間傳送可以大幅降低IO的時間。

再接下來,若我們用系統的time指令來計算時間,我們只能知道包括亂數產生以及排序的real, user, system time,並無法知道個別的資訊。為了要能夠準確亂數產生以及排序個別的資訊,最好的方法是另外自己寫一支測試,然後在每筆測資就fork出一個process來跑主程式,而測試程式除了可以用來更準確的偵測child所花的時間,更可以控制流程、紀錄統計資料,是一個比較適當的方法。

另外,為了使測出來的時間更具有可信度,在同一個資料量,同一個thread數下,我也用不一樣的測資去跑,最後加總平均。這同樣也是可以用自己另外寫的程式完成的。

\section{結果}
以上是我在測試程式上的改進,而對結果造成影響的討論,接下來討論不同測資大小及thread數目下,平行化的merge sort執行的時間結果。(這是把ouput去掉,然後每筆跑五次的加總平均)
\begin{center}
\begin{tabular}{l | l l r r r r r r}
\hline
(sec)& \multicolumn{2}{|l}{\# of threads} & 1 & 2 & 5 & 10 & 25 & 100 \\ \hline
\multirow{5}{*}{Real time}
& $n=10^2$ & $\rightarrow$ & 0.000 & 0.002 & 0.000 & 0.000 & 0.004 & 0.004 \\ \cline{2-9}
& $n=10^4$ & $\rightarrow$ & 0.004 & 0.002 & 0.004 & 0.006 & 0.002 & 0.008 \\ \cline{2-9}
& $n=10^6$ & $\searrow$ & 0.346 & 0.244 & 0.226 & 0.222 & 0.224 & 0.220 \\ \cline{2-9}
& $n=10^7$ & $\searrow$ & 3.800 & 2.634 & 2.310 & 2.286 & 2.264 & 2.278 \\ \cline{2-9}
& $n=5\times10^7$ & $\searrow$ & 20.212 & 13.734 & 11.912 & 11.786 & 11.672 & 11.734 \\
\hline\hline
\multirow{5}{*}{User time}
& $n=10^2$ & $\rightarrow$ & 0.000 & 0.000 & 0.000 & 0.000 & 0.000 & 0.000 \\ \cline{2-9}
& $n=10^4$ & $\rightarrow$ & 0.004 & 0.004 & 0.002 & 0.000 & 0.006 & 0.002 \\ \cline{2-9}
& $n=10^6$ & $\rightarrow$ & 0.340 & 0.344 & 0.344 & 0.336 & 0.338 & 0.326 \\ \cline{2-9}
& $n=10^7$ & $\rightarrow$ & 3.782 & 3.790 & 3.808 & 3.784 & 3.748 & 3.786 \\ \cline{2-9}
& $n=5\times10^7$ & $\rightarrow$ & 20.070 & 20.192 & 20.338 & 20.240 & 20.068 & 19.882 \\
\hline\hline
\multirow{5}{*}{Sys time}
& $n=10^2$ & $\rightarrow$ & 0.000 & 0.000 & 0.000 & 0.000 & 0.000 & 0.000 \\ \cline{2-9}
& $n=10^4$ & $\rightarrow$ & 0.002 & 0.000 & 0.002 & 0.002 & 0.002 & 0.002 \\ \cline{2-9}
& $n=10^6$ & $\rightarrow$ & 0.004 & 0.002 & 0.002 & 0.012 & 0.008 & 0.008 \\ \cline{2-9}
& $n=10^7$ & $\nearrow$ & 0.022 & 0.036 & 0.034 & 0.052 & 0.060 & 0.072 \\ \cline{2-9}
& $n=5\times10^7$ & $\nearrow$ & 0.150 & 0.130 & 0.162 & 0.188 & 0.220 & 0.366 \\
\hline
\end{tabular}
\end{center}

首先我們可以觀察到,在thread為1時,也就是完全沒有平行話時,總時間大概就是user time + system time。而在user time以及system time之中,很明顯地,user time佔了顯著的地位。也因此,即使在n比較大,system time隨著thread數而增加時,real time依舊不會有明顯的增加,反而會因為user time分配在不同處理器下而有更好的表現。

此外,我們可以看到雖然整體而言,實際執行時間隨著thread的增加而減少,但是當thread數太多,例如100時,執行時間反而沒有25來得好,我想,最主要的原因就是system time的增加。雖然效率的降低並不是十分顯著,但是這依舊是可以再仔細考慮,並且trade-off的關鍵。
\end{document}
